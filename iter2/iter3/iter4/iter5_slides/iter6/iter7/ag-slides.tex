\documentclass[pdf]{beamer}
\mode<all>{\usetheme{Warsaw} \useoutertheme{default}}
\mode<handout>{\usecolortheme{seagull}}
%\usenavigationsymbolstemplate{}
\newtheorem{defn}[theorem]{Definition}

\title{Introduction to Commutative Algebra}
\subtitle{and affine algebraic varieties}
\author{Amal M}
\date{\today}

\begin{document}

\AtBeginSection[]{
\begin{frame}{Table of Contents}
	\tableofcontents[currentsection]
\end{frame}
}

\begin{frame}
    \thispagestyle{empty}
    \titlepage
\end{frame}
\addtocounter{framenumber}{-1}

\begin{frame}{Table of Contents}
    \tableofcontents
\end{frame}

%%%%%%%%%%%%%%%%%%%%%%%%%%%%%%%%%%%%%%%%%%%%%%%%%%

\section{Introduction}

\begin{frame}
    \begin{itemize}
        \item Commutative Algebra is the first step in learning Algebraic Geometry. 
        \item Commutative Algebra is the study of algebras over Commutative rings.
        \item Algebraic Geometry is the study of curves given by polynomial equations.
    \end{itemize}
\end{frame}

\begin{frame}
    How is Algebraic Geometry related to Commutative Algebra?
    \begin{itemize}
        \item<2-> Curves given by polynomial equations are examples of algebraic varieties. 
        \item<3-> Algebraic varieties have the properties of both an algebraic object, a $k$-algebra and a geometric object, a curve in space
    \end{itemize}
\end{frame}

\begin{frame}
    How is this useful?
    \begin{itemize}
        \item<2-> Geometry problems can be tackled using algebraic methods
        \item<3-> Algebraic problems can be solved using geometric intuition
        \item<4-> Certain problems in number theory can also be solved!
    \end{itemize}
\end{frame}

\begin{frame}
    Where have we seen this before?
    \pause
     We have seen this in Analytical geometry!
    \pause
    Algebraic geometry may rightly be seen as the modern version of analytical geometry
\end{frame}

\section{Algebraic Variety}

\begin{frame}
    What is an Algebraic Variety?
    \begin{itemize}
        \item<2-> An algebraic variety is the subset of an affine space that is cut out by polynomials.
        \item<3-> It is denoted by V(S)
    \end{itemize}
\end{frame}

\section{Prime and Maximal Spectrum}

\begin{frame}
    Given a commutative ring $A$, $\mathfrak{a}$ is an ideal of the ring if it is a subgroup of $A$ so that if $x \in \mathfrak{a}$ and $a \in A$ then $ax \in \mathfrak{a}$.
    \begin{itemize}
        \item<2-> An ideal $\mathfrak{p}$ is a special ideal with the property such that 
\end{frame}

\section{Zariski Topology}
\section{Localization}
\section{Conclusion}
\end{document}
