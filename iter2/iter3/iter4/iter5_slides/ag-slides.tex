\documentclass[pdf]{beamer}
\mode<all>{\usetheme{Warsaw} \useoutertheme{default}}
\mode<handout>{\usecolortheme{seagull}}
\usenavigationsymbolstemplate{}

\title{Introduction to Commutative Algebra}
\subtitle{and affine algebraic varieties}
\author{Amal M}
\date{\today}

\begin{document}

\AtBeginSection[]{
\begin{frame}{Table of Contents}
	\tableofcontents[currentsection]
\end{frame}
}

\begin{frame}
    \thispagestyle{empty}
    \titlepage
\end{frame}
\addtocounter{framenumber}{-1}

\begin{frame}{Table of Contents}
    \tableofcontents
\end{frame}

%%%%%%%%%%%%%%%%%%%%%%%%%%%%%%%%%%%%%%%%%%%%%%%%%%

\section{Introduction}
\subsection{2 min}


\begin{frame}
    \frametitle{Introduction}
    Talk for two minutes about what you did and introduce yourself
\end{frame}


\section{Algebraic Varieties}
\subsection{3 min}

\begin{frame}
    \frametitle{Curves}
    Introduce elliptic curves as an example of such a curve. Talk about other simple curves first and what kind of shapes their solutions make.
\end{frame}

\begin{frame}
    \frametitle{Polynomial Ring}
    Introduce the polynomial ring, How a single polynomial makes a curve, how a bunch of polynomials is called an ideal 
\end{frame}

\begin{frame}
    \frametitle{Affine Algebraic Varieties}
    Define an affine algebraic variety or set. 
\end{frame}    

\section{Nullstellensatz}
\subsection{5 min}

\begin{frame}
    \frametitle{The Coordinate Ring}
        Define the ideal of a variety and P(X) the coordinate ring
\end{frame}

\begin{frame}
    \frametitle{Nullstellensatz}
    Discuss curves over the complex numbers. If you have such curves then you got the Nullstellensatz which basically gives you a connection between algebra and geometry
\end{frame}

\begin{frame}
    \frametitle{Algebraic - Geometry}
    Thus explain the deep  hidden connection between geometry and algebra
\end{frame}

\begin{frame}
    \frametitle{Regular mappings}
    Explain polynomial mapping/regular mapping between varieties
\end{frame}

\section{Commutative Algebra}
\subsection{2 min}

\begin{frame}
    \frametitle{What sort of Commutative Algebra do we use?}
    What sort of commutative algebra machinery do we use: (Do not explain any of these. Point out where you use them instead)
    \begin{enumerate}
        \item Modules
        \item Tensor products
        \item Exact sequnces
        \item Direct Limits
    \end{enumerate}
\end{frame}

\section{Zariski Topology}
\subsection{3 min}

\begin{frame}
    \frametitle{Zariski}
    Talk about the prime spectrum and the Zariski Topology what sort of machinery would that use? 
\end{frame}

\begin{frame}
    \frametitle{Constructible Topology}
    You can have another topology called the Constructible Topology
\end{frame}

\section{Presheaf and Sheaf}
\subsection{4 min}

\begin{frame}
    \frametitle{Presheaf and Sheaf}
     Definiton of a Presheaf and Sheaf 
 \end{frame}

 \section{Applications}
 \subsection{1 min}

 \begin{frame}
     \frametitle{Applications of Algebraic Geometry}
    Do you really want applications? You could mention in passing string theory, arithmetic geometry, proof of the Fermat’s last theorem etc… 
\end{frame}

\section{Conclusion}

\begin{frame}
    \frametitle{Acknowledgement}
    Hwey Lewis
    Borat
\end{frame}
     
\end{document}
