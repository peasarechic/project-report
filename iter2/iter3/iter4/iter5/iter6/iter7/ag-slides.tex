\documentclass[pdf]{beamer}
\mode<all>{\usetheme{Warsaw} \useoutertheme{default}}
\mode<handout>{\usecolortheme{seagull}}
%\usenavigationsymbolstemplate{}
\newtheorem{defn}[theorem]{Definition}

\title{Introduction to Commutative Algebra}
\subtitle{and affine algebraic varieties}
\author{Amal M}
\date{\today}

\begin{document}

\AtBeginSection[]{
\begin{frame}{Table of Contents}
	\tableofcontents[currentsection]
\end{frame}
}

\begin{frame}
    \thispagestyle{empty}
    \titlepage
\end{frame}
\addtocounter{framenumber}{-1}

\begin{frame}{Table of Contents}
    \tableofcontents
\end{frame}

%%%%%%%%%%%%%%%%%%%%%%%%%%%%%%%%%%%%%%%%%%%%%%%%%%

\section{Introduction}

\begin{frame}
    \frametitle{Overview}
    Paraphrase the intro from the project report
\end{frame}

\section{Algebraic Variety}

\begin{frame}
    \frametitle{Algebraic Variety}
    You still don't understand the connection very well. So you got to revise first before you can write this.
    Ask the question. How is every algebraic variety a nilpotent free k-algebra? How is $Spec(A)$ the solution set of polynomials?
    How is Spec(A) a variety? You only shown that it can have a Topology. How is it an algebraic variety? 
    Reread and understand the dictionary in Vakil's notes in full.
\end{frame}

\section{Prime and Maximal Spectrum}

\begin{frame}
    \frametitle{Prime Ideal}
    How can you make this more intuitive?
\end{frame}

\section{Zariski Topology}
\section{Localization}
\section{Conclusion}
\end{document}
