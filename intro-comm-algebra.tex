\documentclass[]{report}
\usepackage{amsmath, amsthm, amssymb, amsfonts}
\usepackage[a4paper, total={3in, 8in}]{geometry}


% user defined commands go here
\newtheorem{theorem}{Theorem}[section]
\newtheorem{corollary}{Corollary}[theorem]
\newtheorem{lemma}[theorem]{Lemma}

\begin{document}

\title{Introduction to Commutative Algebra}
\author{Amal M}
\date{January 4, 2021}
\maketitle

\begin{abstract}
    The motivation for the study of algebraic geometry is how algebraic objects (rings of rational functions) are associated with varieties (zeros of polynomials). This subject florished during the second half of the twentieth century. Algebraic geometry allows us to study the geometry arising from algebraic objects. Core to the deeper understanding of this subject is an understanding of the subject of commutative algebra which studies commutative rings and their ideals and modules. The purpose of the present project is to gain an understanding of commutative algebra through solving exercises from Atiyah-MacDonald's book, Introduction to Commutative Algebra. The reading project comprised of the study of the theory of rings and modules, their tensor product and exact sequences of rings and modules. The project concluded with a proof of the Going-Up Theorem.

\end{abstract}

\chapter{Introduction: What is Algebraic Geometry?} % this chapter contains no math beyond the statement of the problems
\section{Some historical problems}
\subsection{27 lines}
\subsection{Bezout's Theorem}
\subsection{Brief History of Algebraic Geometry}
\section{What is geometry?}
\section{What is algebra?}




\chapter{Hilbert's Nullstellensatz}
\section{Basics ideas in ring theory} % provide some basic definitions and motivate the study of the nullstellensatz showing the connection between geometry and algebra
\section{The Nullstellensatz}

% The statement of the theorems and proofs go here. Focus most effort here.
\chapter{Rings and Ideals}
\section{Basic Definitions}
\begin{theorem}
    There is a one-to-one order-preserving correspondence between the ideals \(\mathfrak{b}\) of $A$ which contain \(\mathfrak{a}\), and the ideals $\bar\mathfrak{b}$ of $A/\mathfrak{a}$, given by $\mathfrak{b} = \phi^{-1}(\mathfrak{b})$. 
\end{theorem}

\subsection{Prime and Maximal Ideals}
An ideal $\mathfrak{p}$ in $A$ is prime if $\mathfrak{p} \neq (1)$ and if $xy\in \mathfrak{p}\implies x\in \mathfrak{p} \text{ or } y\in \mathfrak{p}$.

An ideal $\mathfrak{m}$ in $A$ is maximal if $\mathfrak{m}\neq (1)$ and if there is no ideal $\mathfrak{a}$ such that $\mathfrak{m\subset a}\subset (1)$. Equivalently,
$$\mathfrak{p} \text{ is prime } \Leftrightarrow A/\mathfrak{p} \text{ is an integral domain}$$
$$\mathfrak{m} \text{ is maximal} \Leftrightarrow A/\mathfrak{m} \text{ is a field}$$

\begin{theorem}
    Every ring $A\neq 0$ has at least one maximal ideal.
\end{theorem}

\subsection{Nilradical and Jacobson Radical}

The set $\mathfrak{N}$ of all nilpotent elements of a ring $A$ is an ideal called the nilradical. The nilradical of $A$ is the intersection of all the prime ideals of $A$.

The Jacobson radical $\mathfrak{R}$ of $A$ is defined to be the intersection of all the maximal ideals of $A$. It can be characterized as follows:

\begin{theorem}
    $x\in \mathfrak{R} \Leftrightarrow 1-xy$ is a unit in $A$ for all $y\in A$.
\end{theorem}

\section{Exercises}
\subsection{The prime spectrum of a ring}

We denote by $X = Spec(A)$ the set of all prime ideals of the ring $A$. This is called the prime spectrum of the ring $A$. If we think of each prime ideal $\mathfrak{p}_x$ as a point $x\in X$ then we can associate a topology to the space $X$ called the Zariski Topology.

Let $E\subset A$ be a subset of $A$. Then let $V(E)$ denote the set of all prime ideals of A which contain E.

\textbf{Exercises 16.} Prove that

i) If $\mathfrak{a}$ is the ideal generated by $E$, then $V(E) = V(\mathfrak{a}) = V(r(\mathfrak{a}))$.

ii) $V(0) = X, V(1) = \varnothing$.

iii) if $(E_i)_{i\in I}$ is any family of subsets of A, then
$$V(\cup_{i\in I} E_i) = \cap_{i\in I} V(E_i)$$
iv) $V(\mathfrak{a\cap b}) = V(\mathfrak{ab}) = V(\mathfrak{a}) \cup V(\mathfrak{b})$ for any ideal $\mathfrak{a,b}$ of $A$.

\chapter{Modules}

\chapter{Rings and Modules of Fractions}

\chapter{Integral Dependence}

\begin{thebibliography}{9}
    % type bibliography here
\end{thebibliography}

\end{document}
